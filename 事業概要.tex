% latexmk -pvc 事業概要.tex
\documentclass[uplatex, a4j]{jsarticle}
\usepackage[a4paper]{geometry}
\usepackage[dvipdfmx]{graphicx}
\usepackage{pxrubrica}

\date{\today}
\title{SFC-LT ~福澤諭吉で学問を奨める社会システム~}
\author{湘南藤沢Projects 代表 宮元 眺}

\begin{document}
\newgeometry{top=2cm,bottom=2cm,left=2cm,right=2cm}
% \newgeometry{}


\maketitle
\begin{abstract}
  本書ではSFCの学生を取り巻く「奨学金」,「大学への寄付」,「SFC」の3つについての問題の提示と,それらを解決するための社会システム「SFC-LT」の提案を行う.「SFC-LT」はソーシャルメディアとクラウドファンディングを利用した学生の意欲に応じて支援総額がスケールする奨学金システムである.年間100人のSFCの学生に謝礼金1万円で3分講演してもらうことで,先の3つについての問題をすべて解決できると期待される.

\end{abstract}

\section{3つの問題}

\subsection{「奨学金」の問題}
  ここでの奨学金とは「学問を奨めるためのお金」であり,奨励制度や研究費などの学生の学びを支援する制度に関しても奨学金としてあつかう.

  現状の奨学金は貸与型と給付型の2種類に分けられるがどちらも問題を抱えている.
  前者は奨学金とは名ばかりの体のいい借金にすぎず,リスクコントロールという面で学生の選択肢を増やすことはできても,「学び」への明確なメリットにはつながらない.結果として「学位を取るためだけのお金」と成り果てている.
  後者については応募数に応じて総額が変動しないため,1つのパイの奪い合いにすぎず,供給がスケールしない.100人応募しようが1000人応募しようが,給付される人数は変わらないわけである.
  また,両者の奨学金について情報が伝播しづらく潜在的な学ぶ意欲のある学生\footnote{本来,学生は学ぶ意欲のあるものであるが学ばない学生もいる現状を踏まえて}に存在を知られづらいという問題もある.


\subsection{「大学への寄付」の問題}
  大学への寄付に関しても問題が2つあげられる.
  1つは寄付の使い道を寄贈者が選択することができない点である.
  例えば,弊学ではSFCの卒業生が大学に対して寄付を行っても大半が三田や日吉の方に流れてしまう.\footnote{メンター三田会 代表理事 宮地さん談}
  もう1つは,透明性が低く,寄付が何に使われているのかわからない点である.
  2012年から16年にかけて未来創造塾建設の際に慶應義塾大学は寄付を募集\footnote{噂では数億円が集まったらしい.}したが,2018年11月現在,未来創造塾の建設は進んでいない.代わりに建設予定地の南方にSBC\footnote{Student Build Campus.学生自らが設計・建築・場作りに携わるキャンパス.}と呼ばれる滞在型教育施設ができているが,それについて大学からの公式な情報公開はなされていない\cite{mirai}.


% \subsection{「大学経営」の問題}
%   大学経営は外部環境の変化により今までにない危機に直面している.通信技術の進歩と導入コストの減少により講義動画配信とインタラクティブな議論が可能になり,「MOOC\footnote{Massive Open Online Course.インターネットを通じて配信される,無償または安価で受講できる講義}\cite{mooc}」や「ミネルバ大学」,オンラインサロンなどの競合が出現し始めた.
%   既存の大学に比べ低価格で高密度なこれらの競合サービスに対し,補助金のために文部科学省の方針に縛られ内部環境の変化を行えない大学\cite{kakuitsu}は,今後これまでにない経営上の困難さに直面することが予想される.
%
%
% \subsection{「企業・団体の人材採用」の問題}
%   工業化時代が終焉を迎え人員の増加に比例して生産性が伸びなくなり,少人数でも大量生産が行える時代へと加速度的に変化している.
%   問題解決能力の高い人材への需要が高まっているにも関わらず,我が国の多くの高等教育期間はその需要に対応できておらいない.なぜなら我が国の多くの入試制度は学力の偏差値を唯一の判断基軸としているために,学内の学生は自ずと画一化されてしまい\cite{kakuitsu},その環境下では問題解決において重要な「多様性を受け入れる」\cite{harvard}という要素を育めないためである.


\subsection{「慶應大学湘南藤沢キャンパス」の問題}
  この事態を予見し画一化された高等教育からの脱却のためにつくられたSFC\cite{kakuitsu}でさえ,その多様性故の問題を抱えている.
  AO入試などの導入によりキャンパス内には多種多様な学生がいるが,昼休みのない時間割や必修講義がほぼ存在しないことなどが原因となって他の学生と同じ時間を過ごす機会が少なくなり横の繋がりが非常に弱い.
  SFCの講義の一環として催される「政策コーカス」\footnote{SFC
  生が入学してはじめに受講する「総合政策学」の中で催される政策コンテスト.総合政策学部の必修の講義のため500名程度が参加.}では,毎年,このSFC生同士の繋がりの薄さを問題視し解決を試みるチームが出現するほどである.ただ,毎年,同じ問題に挑んでいるにも限らず,縦の繋がりが弱いためにゼロから同じことをして,その殆どが持続性が無く失敗\footnote{かくいう筆者も2016年度にこの問題に挑み失敗した身である.}に終わっている.

  SFC自体が誕生してから30年程度と比較的若いため,SFC三田会も事実上機能していない.\cite{diamond}
  % 縦と横の繋がりを作ろうとするプラットフォームを作ろうとしても,プラットフォームが無いために過去の遺産にアクセスできずに,ゼロから作り出そうとして頓挫し続けているのが現状である.
  また,あまりにも学べる学問領域が広すぎるために何も興味を見つけられずに卒業していく学生も多い.



\section{提案}
\subsection{SFC-LT}

これらの問題の解決のために,ソーシャルメディア\footnote{TwitterやFacebookなどのSNSでの拡散を利用して情報発信を行うメディア}とクラウドファンディング\footnote{インターネットを利用した資金調達}を複合した奨学金システム(通称SFC-LT)を提案する.

具体的には,年間100人程度の学生に講演料1万円を渡して3分間,その学生の取り組んでいることや取り組みたいことを講演してもらう.その講演の動画をTwitterなどのSNSを用いて配信・拡散していき,同時にその学生への支援をクラウドファンディングする.ソーシャルメディアの運営費用は,卒業した講演者からの寄附,学生の採用を狙う企業・団体からのスポンサード,クラウドファンディングの手数料の3つの収入でまかなう.「3分で1万円が稼げるから」という明確なインセンティブに加え,SFCの入学者の約10\%にあたる100人を講演させることで「みんなやってるから」という「言い訳」により,普段,周囲を気にして行動できない学生をも巻き込める.キャンパス中の人が参加できる環境を作り出せば,その後は以下のロジックで先の3つの問題すべてを解決できる.


\subsection{問題解決のロジック}
  \begin{enumerate}
    \item 3分1万円の条件により安定的に100名の学生が講演を行う.
    \item 講演者が関係者となるためこの社会システムを取り巻くコミュニティは毎年100人づつ拡大する.
    \item コミュニティの拡大によりソーシャルメディアの拡散力が上がる.
    \item 拡散力が上がれば潜在的な支援者に情報がリーチし,支援総額も上がり手数料による収益が増えて①が安定化.
      \begin{enumerate}
        \item 3分1万円という割高な条件は学ぶことへの直接的なインセンティブとなり,講演者が増えれば増えるほど情報が拡散することで支援総額もスケールする.同時に,この奨学金システムの情報自体も拡散される.(\textbf{「奨学金」の問題が解決})
        \item 寄贈者は自分の支援したい学生を選んで寄付を送ることができ,その学生の講演自体を聞いた上で判断ができるので情報の透明性も保てる.(\textbf{「大学への寄付」の問題が解決})
      \end{enumerate}
    \item 講演の質が低い学生も出るが,それにより非活動的な学生が講演をする心理的な障壁が下がる.
    \item 心理的な障壁が下がったことで講演希望者が増える.
      \begin{enumerate}
        \item SNSを通じて知人の講演動画が流れてくることで他者の興味を知れて
        \item 同時に視聴している学生自身の興味が見つかる可能性が増える.
        \item 学生の活動を記録するプラットフォームとして過去の学生の活動を知れるようになる.
        \item その人にコンタクトを取ることで過去の遺産を引き継げるようになり縦の繋がりもできる.(\textbf{「SFC」の問題が解決})
      \end{enumerate}
    % \item 講演希望者が増えても講演枠は100人と限られるため競争が激化し講演者の質が向上.
    % \item 講演の質が上がればSFCの認知度が上がり入学志願者が増える.
    % \item 入学志願者が増えれば競争が激化し入学生の質が向上します.
    %   \begin{enumerate}
    %     \item 入学→講演→卒業→支援の流れができることで縦の繋がりが強固なものとなりコミュニティは縦に広がりだす.
    %     \item 縦の広がりは歴史ある慶應大学を遡って拡大していき,新興的なサロンなどの競合では真似できない優位性を発揮する.(\textbf{「大学経営」の問題が解決})
    %   \end{enumerate}
    % \item 入学生の質が上がれば,学生全体の質が向上し社会への影響力も増加する.
    % \item 学生全体の質が向上し社会への影響力が増せば,企業や団体がSFCの学生を欲しがる.
    % \item 企業や団体がSFCの学生を欲しがると,スポンサードが強くなり社会のリソースがSFCに流れ込む
    % \item 以上のロジックが通りこの社会実験がSFCで成功すれば,それをロールモデルとして他大学でも同様の試みが行われるようになる.
    %   \begin{enumerate}
    %     \item これにより各大学で社会と学生との距離が近づき,社会のニーズに対応した人材を供給できるキャンパスが増えていく.(\textbf{「企業・団体の人材採用」の問題が解決})
    %   \end{enumerate}
  \end{enumerate}

  \begin{center}
    \includegraphics[width=1\linewidth]{ステークホルダー関係図.png}
  \end{center}


\begin{thebibliography}{数字}
  \bibitem{mirai} 「アーカイブ | 慶應義塾大学 未来創造塾」 http://www.miraisozo.sfc.keio.ac.jp/news/
  \bibitem{what_is_mooc} 「教育の最先端:MOOCって何? その利用法と実践ノウハウまで」 https://www.highedu.kyoto-u.ac.jp/connect/topics/report01/1\_ws\_flyer161017.pdf
  \bibitem{mooc} "mooc.org | Massive Open Online Courses | An edX Site" http://mooc.org/
  \bibitem{kakuitsu} 「新時代の高等教育を考える―画一的教育からの脱却」 加藤 寛 1986/10
  \bibitem{harvard} "Teams Solve Problems Faster When They’re More Cognitively Diverse" Alison Reynolds, David Lewis 2017/3 https://hbr.org/2017/03/teams-solve-problems-faster-when-theyre-more-cognitively-diverse
  \bibitem{diamond} 「週刊ダイヤモンド 2016年5/28号」 ダイヤモンド社 2016/5
\end{thebibliography}

% \section{チーム体制}
%   NPO法人設立までの初期段階では,以下のチーム体制で事業の準備に臨む.
%   \begin{itemize}
%     \item 宮元眺(代表)…チームの意思決定とマネジメント,プロダクトデザインを行う.
%     \item 醍醐旭裕(監事)…NPO法人設立の手続きと大学での活動にあたっての学事との調整を行う.
%     \item 鈴木博子(職員)…その他の必要な事務作業などを担当する.
%     \item Ruskinics(株式会社の設立予定の外部団体) … ソーシャルメディアやクラウドファンディングなどの必要なシステム開発を行い,ASP契約を結んで当団体で利用する.
%   \end{itemize}
%
% \restoregeometry
% \newgeometry{left=2cm,right=2cm}

\end{document}
